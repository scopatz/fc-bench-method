\section{Dynamic Time Warping}
\label{dtw}

Now that there are representative models of all features, the issue revolves 
around how to compare these time series models. Dynamic time warping \cite{muller}
is a method for computing the distance between any two time series. The time 
series need not be of the same length.  Furthermore, the time series may be 
decomposed into a set of representative spectra and the DTW may still be applied.
The distance computed by DTW is a measure of the changes that would need to be 
made to one time series to turn it into (warp) the other time series. Thus, 
the DTW distance is a measure over the whole time series, and not just a 
single characteristic point.

With respect to nuclear feul cycle benchmarks, there are two main DTW applictions.
The first is to compute the distance between a Gaussian process model and each of 
the simulators that made up the training set.  This gives a quantitative measure 
of how far each simulator is from the model and can help determine which 
simultors are outliers. To maintain a non-judgemental benchmark, though, it is 
critical to not then use this information to discard outliers.  Rather, outlier
identification should be used as part of an inter-code comparison. If one simulator
is an outlier for a given metric, the reasons for this should be investigated. 
For example, the outlier simulator may be at higher fidelity level, there may be 
a bug in the outlier, or there may be a bug in all other simulators. Identifying 
outliers for many metrics could help discover the underlying cause of any 
discrepencies.

The second application of DTWs to benchmarking is to compare the consituient 
feature models to the total model.  Thes distances computed in this way allow 
for a rank ordering of the componets.  This enables the benchmark to make claims
about which features drive the fuel cycle metric most strongly, over the whole 
simulation time for all simulators. Traditionally, the simulators have to agree
within nominal error bounds ($<5\%$) for a benchmark to make such a claim.  Here,
the simulators need not necessarily agree since the Gaussian process models are
used as representatives.  In this application, it is useful to recast the DTW 
distance as a measure of contribution.  This new FOM will be presented in 
\S\ref{contribution}.

For any two time series, dynamic time warping consists of three part:
the distance $d$, a cost matrix $C$, and a warp path $w$. The cost matix 
specifies how far a point on the first time series is from another point on the 
other time series.  The warp path is then the minimal cost curve through this 
matrix from the fist point in time to the last. The distance, therefore, is the
total cost of traversing the warp path.
