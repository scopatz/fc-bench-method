\documentclass{ntmanuscript}
%\usepackage[acronym,toc]{glossaries}
%\include{acros}
%\makeglossaries
%%%%%%%%%%%%%%%%%%%%%%%%%%%%%%%%%%%
\title{Non-judgemental Dynamic Fuel Cycle Benchmarking}

% Authors. Separated by commas
\author{Anthony Micahel Scopatz$^1$}

% Institutes of the authors
\input{institutions}

% Information concerning the person submitting the manuscript
\submitter{Anthony M. Scopatz}
\submitteraddress{541 Main Street, Columbia, SC 29208}
\submitteremail{scopatz@cec.sc.edu}


% No more than three keywords, though each can be a phrase
\keywords{fuel cycle, gaussian process models, dynamic time warping}

\usepackage{color}
\usepackage{graphicx}
\usepackage{booktabs} % nice rules for tables
\usepackage{microtype} % if using PDF
\usepackage{xspace}
\usepackage{listings}
\usepackage{textcomp}
\usepackage{ulem}

\definecolor{listinggray}{gray}{0.9}
\definecolor{lbcolor}{rgb}{0.9,0.9,0.9}
\lstset{
    %backgroundcolor=\color{lbcolor},
    language={C++},
    tabsize=4,
    rulecolor=\color{black},
    upquote=true,
    aboveskip={1.5\baselineskip},
    belowskip={1.5\baselineskip},
    columns=fixed,
    extendedchars=true,
    breaklines=true,
    prebreak=\raisebox{0ex}[0ex][0ex]{\ensuremath{\hookleftarrow}},
    frame=single,
    showtabs=false,
    showspaces=false,
    showstringspaces=false,
    basicstyle=\scriptsize\ttfamily\color{green!40!black},
    keywordstyle=\color[rgb]{0,0,1.0},
    commentstyle=\color[rgb]{0.133,0.545,0.133},
    stringstyle=\color[rgb]{0.627,0.126,0.941},
    numberstyle=\color[rgb]{0,1,0},
    identifierstyle=\color{black},
    captionpos=t,
}

\newcommand{\code}[1]{\lstinline[basicstyle=\ttfamily\color{green!40!black}]|#1|}
\newcommand{\units}[1] {\:\text{#1}}%
\newcommand{\SN}{S$_N$}
\newcommand{\cyclus}{\textsc{Cyclus}\xspace}
\newcommand{\Cyclus}{\cyclus}
\newcommand{\citeme}{\textcolor{red}{CITE}\xspace}
\newcommand{\cycpp}{\code{cycpp}\xspace}
\newcommand{\TODO}[1] {{\color{red}\textbf{TODO: #1}}}%

\newcommand{\comment}[1]{{\color{green}\textbf{#1}}}


\date{}
%%%%%%%%%%%%%%%%%%%%%%%%%%%%%%%%%%%
\begin{document}

\begin{abstract}
My happy abstract.
\end{abstract}

\section{Introduction}
\label{intro}
The act of fuel cycle benchmarking has long faced methodological issues 
on what metrics to compare, how to compare them, and at what point in the
fuel cycle they should be compared. This is partly because such activities 
are not benchmarking in the strictess validation sense. Most fuel
cycle benchmarks are more correctly called code-to-code comparisons or 
inter-code comparisons, as they compare simultor results. Importantly, 
these take place in the absence of true experimental data. The number of 
real-world, industrial scale nuclear fuel cycles that have historically been 
deployed is not sufficient for statistical accuracy even for the Once-Through 
sceanario. For other fuel cycles, industrial data is even more stark. 
Since fuel cycle simulation is thus effectively impossible to validate, 
we should look to methods non-judgemental methods of benchmarking. The 
results of any given simulator should be evaluated in reference to how 
it performs against other simulators in such a way that acknowledges that 
any and all simulators may demonstrate incorrect behaviour. No simulator
by fiat produces the `true' or reference case.

The other major conceptual issue with fuel cycle benchmarking is that there 
is no agreed upon mechanism for establishing a figure-of-merit (FOM) for 
a metric that is uniform across all metrics of interest. For example, 
repository heat load may be examined only at the end of the of the simulation,
separatred plutonium may be used whereever it peaks, and natural uranium 
mined might be of concern only in 100 years. Comparing at a specific point 
in time fails to take into account the behaviour of that metric over time and 
can skew any decisons made based soley on that metric. Additionally, the 
time of comparison varies based on the metric itself. This is a necessary 
side effect of picking a single point in time.
Furthermore, such FOMs are not useful for indicating why simulations differ, 
only that they do. Moroever, if such FOM match, this does not indicate
that the simulator actually agree. Their behaviour could be radically 
different at every other point in time.  It should be noted that 
Equillibrium and quasi-static fuel cycle simulators are sometimes able to 
ignore these issues, because all time points are treated equally.

Some dynamic FOMs do exist. 



Such as cyclus \cite{cyclus_v1_2}.
%\input{methods}
%\input{implementation}
%\section{Conclusions \& Future Work}
\label{conclusion}

This paper demonstrates a robust method for generating figures-of-merit
for nuclear fuel cycle benchmarking activities by coupling Gaussian process
regression to dynamic time warping. This method takes advantage of modeling
uncertainties in fuel cycle metrics if they are known. It is also capable 
of handling the situation where different simulators output metric data on
vastly different time grids. The distance computed by the dynamic time 
warping can itself serve as the figure-of-merit. Additionally, the 
distance can also be used to derive contribution and normalized contribution
figures-of-merit. These measures are valuable for non-judgementally 
determining the impact of different constiuent measures to a total 
metric, e.g. LWR versus FR power generation. The contribution metrics also 
scale such that higher values imply higher impact. This is sometimes
more intuitve as compared to a DTW distance, where lower values imply
more similar curves.

Any regression method could have been used to form a model. Similarly, any
mechanism for comparing two time series could have been used as a measure
of distance.  However, Gaussian processes and DTW were chosen because of 
the nature of a benchmarks and inter-code comparisons that lack experimental
validation. It is not possible to build out a given fuel cycle scenario
and see how it performs 200 years in the future. Furthermore, using 
historical data for validation provides too few cases for comparison and 
each simulator could simply be tuned to precisely match historical events.
Thus, each simulator in a benchmark could be valid or they all could be 
invalid. It is therefore necessary for the FOM to not skew for or against 
any particular simulator. Gaussian process models as used here do not 
judge the simulators differently. The DTW then takes into account the 
cumulative effect of the whole time domain and does not preferentially 
select certain times.

The sample benchmark presented here was very simple and was used for motivation 
purposes only. It consisted of just
two simulators (DYMOND and Cyclus) and one metric (generated power) with
two components (LWR and FR).  However, both Gaussian processes and DTW
are inherently multivariate. More complex forms of analysis could therefore
be performed. For example, the Gaussian process could jointly model the 
effect from many inputs onto the metric. Perhaps the benchmark is formulated
to look at the generated power as a function of time and the power demand curve.
In this case, a two dimensional GP model would be used. Alternatively, 
suppose that a matrix time series of the all individual nuclide mass flows 
are available. DTW is still able compute the distance between two 
such matrices. This would yield a measure of how the mass flows themselves
differ - taking into account each nuclide component - without rely on a collapsed
one dimensional total mass flow curve.  Such cases will be considered in
future work as real inter-code comparison data becomes available.

Furthermore, this work focused on the particular use case of benchmarking.
However, the FOM calculations presented here could also be used to evaluate 
different fuel cycle scenarios. DTW distances could be computed between
a business-as-usual once through scenario and an LWR-to-FR transition
scenario, or any other proposed scenario. This provides a measure for 
comparing the relative cost (in units of the metric, not necessarily 
economic) for selecting one cycle over another. The work here, thus, 
should be seen as a stepping stone to further fuel cycle scenario evaluation
work.

Lastly, dynamic time warping could itself serve a purpose as the objective 
function in a fuel cycle optimization.  For example, suppose a power demand curve 
such as 1\% power growth is known. The DTW distance from the total generated
power to this curve could be minimized as a function of the reactor 
deployment schedule. Such a distance could potentially yield a more 
precise or faster optimization process than simply taking the sum of 
the differences between two time series. Such an optimization would also allow
for matching on multiple time series features simultaneously while retaining
a real-valued objective function.



%%%%%%%%%%%%%%%%%%%%%%%%%%%%%%%%%%%%%%%%%%%%%%%%%%%%%%%%%%%%%%%%%%%%%%%%%%%%%%%%
\bibliographystyle{ans}
\bibliography{refs}
\end{document}
