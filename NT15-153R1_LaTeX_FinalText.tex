\documentclass{ntmanuscript}
%\usepackage[acronym,toc]{glossaries}
%\include{acros}
%\makeglossaries
%%%%%%%%%%%%%%%%%%%%%%%%%%%%%%%%%%%
\title{Non-judgemental Dynamic Fuel Cycle Benchmarking}

% Authors. Separated by commas
\author{Anthony Michael Scopatz$^1$}

% Institutes of the authors
\institute{$^1$University of South Carolina, Department of Mechanical
    Engineering, Nuclear Engineering Program, Columbia, SC 29201}

% Information concerning the person submitting the manuscript
\submitter{Anthony M. Scopatz}
\submitteraddress{541 Main Street, Columbia, SC 29208}
\submitteremail{scopatz@cec.sc.edu}


% No more than three keywords, though each can be a phrase
\keywords{nuclear fuel cycle, gaussian process, dynamic time warping}

\usepackage{color}
\usepackage{graphicx}
\usepackage{booktabs} % nice rules for tables
\usepackage{microtype} % if using PDF
\usepackage{xspace}
\usepackage{listings}
\usepackage{textcomp}
\usepackage[normalem]{ulem}
\usepackage{amssymb}
\DeclareMathAlphabet{\mathpzc}{OT1}{pzc}{m}{it}

\definecolor{listinggray}{gray}{0.9}
\definecolor{lbcolor}{rgb}{0.9,0.9,0.9}
\lstset{
    %backgroundcolor=\color{lbcolor},
    language={C++},
    tabsize=4,
    rulecolor=\color{black},
    upquote=true,
    aboveskip={1.5\baselineskip},
    belowskip={1.5\baselineskip},
    columns=fixed,
    extendedchars=true,
    breaklines=true,
    prebreak=\raisebox{0ex}[0ex][0ex]{\ensuremath{\hookleftarrow}},
    frame=single,
    showtabs=false,
    showspaces=false,
    showstringspaces=false,
    basicstyle=\scriptsize\ttfamily\color{green!40!black},
    keywordstyle=\color[rgb]{0,0,1.0},
    commentstyle=\color[rgb]{0.133,0.545,0.133},
    stringstyle=\color[rgb]{0.627,0.126,0.941},
    numberstyle=\color[rgb]{0,1,0},
    identifierstyle=\color{black},
    captionpos=t,
}

\newcommand{\code}[1]{\lstinline[basicstyle=\ttfamily\color{green!40!black}]|#1|}
\newcommand{\units}[1] {\:\text{#1}}%
\newcommand{\SN}{S$_N$}
\newcommand{\cyclus}{\textsc{Cyclus}\xspace}
\newcommand{\Cyclus}{\cyclus}
\newcommand{\citeme}{\textcolor{red}{CITE}\xspace}
\newcommand{\cycpp}{\code{cycpp}\xspace}
\newcommand{\TODO}[1] {{\color{red}\textbf{TODO: #1}}}%

\newcommand{\comment}[1]{{\color{green}\textbf{#1}}}

\newcommand{\E}{\mathbb{E}}
\newcommand{\GP}{\mathpzc{GP}}
\newcommand{\LWR}{\mathrm{LWR}}
\newcommand{\FR}{\mathrm{FR}}
\newcommand{\Total}{\mathrm{Total}}
\newcommand{\argmin}{\mathrm{argmin}}
\newcommand{\CYCLUS}{\mathrm{Cyclus}}
\newcommand{\DYMOND}{\mathrm{DYMOND}}
\newcommand{\I}{\mathbf{I}}
\newcommand{\K}{\mathbf{K}}


\date{}
%%%%%%%%%%%%%%%%%%%%%%%%%%%%%%%%%%%
\begin{document}

\begin{abstract}
This paper presents a new fuel cycle benchmarking analysis methodology
by coupling Gaussian process regression, a popular technique in Machine
Learning, to dynamic time warping, a mechanism widely used in speech
recognition. Together they generate figures-of-merit for a suite of fuel
cycle realizations. The figures-of-merit may be computed for any time
series metric that is of interest to a benchmark. For a given metric,
these figures-of-merit have the advantage that they reduce the
dimensionality to a scalar and are thus directly comparable.
The figures-of-merit
account for uncertainty in the metric itself, utilize information
across the whole time domain, and do not require that the simulators
use a common time grid. Here, a distance measure is defined that can be used
to compare the performance of each simulator for a given metric. Additionally,
a contribution measure is derived from the distance measure that can be used
to rank order the impact of different partitions of a fuel cycle metric.
Lastly, this paper
warns against using standard signal processing techniques for error reduction,
as error reduction is better handled by the Gaussian process regression
itself.
\end{abstract}

\input{intro}
\input{setup}
\input{gp}
\input{dtw}
\input{contribution}
\input{filtering}
\input{conclusion}
\input{acknow}

%%%%%%%%%%%%%%%%%%%%%%%%%%%%%%%%%%%%%%%%%%%%%%%%%%%%%%%%%%%%%%%%%%%%%%%%%%%%%%%%
\bibliographystyle{ans}
\bibliography{refs}
\end{document}
