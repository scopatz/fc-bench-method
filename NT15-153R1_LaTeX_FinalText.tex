\documentclass{ntmanuscript}
%\usepackage[acronym,toc]{glossaries}
%\include{acros}
%\makeglossaries
%%%%%%%%%%%%%%%%%%%%%%%%%%%%%%%%%%%
\title{Non-judgemental Dynamic Fuel Cycle Benchmarking}

% Authors. Separated by commas
\author{Anthony Michael Scopatz$^1$}

% Institutes of the authors
\institute{$^1$University of South Carolina, Department of Mechanical
    Engineering, Nuclear Engineering Program, Columbia, SC 29201}

% Information concerning the person submitting the manuscript
\submitter{Anthony M. Scopatz}
\submitteraddress{541 Main Street, Columbia, SC 29208}
\submitteremail{scopatz@cec.sc.edu}


% No more than three keywords, though each can be a phrase
\keywords{nuclear fuel cycle, gaussian process, dynamic time warping}

\usepackage{color}
\usepackage{graphicx}
\usepackage{booktabs} % nice rules for tables
\usepackage{microtype} % if using PDF
\usepackage{xspace}
\usepackage{listings}
\usepackage{textcomp}
\usepackage[normalem]{ulem}
\usepackage{amssymb}
\DeclareMathAlphabet{\mathpzc}{OT1}{pzc}{m}{it}

\definecolor{listinggray}{gray}{0.9}
\definecolor{lbcolor}{rgb}{0.9,0.9,0.9}
\lstset{
    %backgroundcolor=\color{lbcolor},
    language={C++},
    tabsize=4,
    rulecolor=\color{black},
    upquote=true,
    aboveskip={1.5\baselineskip},
    belowskip={1.5\baselineskip},
    columns=fixed,
    extendedchars=true,
    breaklines=true,
    prebreak=\raisebox{0ex}[0ex][0ex]{\ensuremath{\hookleftarrow}},
    frame=single,
    showtabs=false,
    showspaces=false,
    showstringspaces=false,
    basicstyle=\scriptsize\ttfamily\color{green!40!black},
    keywordstyle=\color[rgb]{0,0,1.0},
    commentstyle=\color[rgb]{0.133,0.545,0.133},
    stringstyle=\color[rgb]{0.627,0.126,0.941},
    numberstyle=\color[rgb]{0,1,0},
    identifierstyle=\color{black},
    captionpos=t,
}

\newcommand{\code}[1]{\lstinline[basicstyle=\ttfamily\color{green!40!black}]|#1|}
\newcommand{\units}[1] {\:\text{#1}}%
\newcommand{\SN}{S$_N$}
\newcommand{\cyclus}{\textsc{Cyclus}\xspace}
\newcommand{\Cyclus}{\cyclus}
\newcommand{\citeme}{\textcolor{red}{CITE}\xspace}
\newcommand{\cycpp}{\code{cycpp}\xspace}
\newcommand{\TODO}[1] {{\color{red}\textbf{TODO: #1}}}%

\newcommand{\comment}[1]{{\color{green}\textbf{#1}}}

\newcommand{\E}{\mathbb{E}}
\newcommand{\GP}{\mathpzc{GP}}
\newcommand{\LWR}{\mathrm{LWR}}
\newcommand{\FR}{\mathrm{FR}}
\newcommand{\Total}{\mathrm{Total}}
\newcommand{\argmin}{\mathrm{argmin}}
\newcommand{\CYCLUS}{\mathrm{Cyclus}}
\newcommand{\DYMOND}{\mathrm{DYMOND}}
\newcommand{\I}{\mathbf{I}}
\newcommand{\K}{\mathbf{K}}


\date{}
%%%%%%%%%%%%%%%%%%%%%%%%%%%%%%%%%%%
\begin{document}

\begin{abstract}
This paper presents a new fuel cycle benchmarking analysis methodology
by coupling Gaussian process regression, a popular technique in Machine
Learning, to dynamic time warping, a mechanism widely used in speech
recognition. Together they generate figures-of-merit for a suite of fuel
cycle realizations. The figures-of-merit may be computed for any time
series metric that is of interest to a benchmark. For a given metric,
these figures-of-merit have the advantage that they reduce the
dimensionality to a scalar and are thus directly comparable.
The figures-of-merit
account for uncertainty in the metric itself, utilize information
across the whole time domain, and do not require that the simulators
use a common time grid. Here, a distance measure is defined that can be used
to compare the performance of each simulator for a given metric. Additionally,
a contribution measure is derived from the distance measure that can be used
to rank order the impact of different partitions of a fuel cycle metric.
Lastly, this paper
warns against using standard signal processing techniques for error reduction,
as error reduction is better handled by the Gaussian process regression
itself.
\end{abstract}

\section{Introduction}
\label{intro}
The act of fuel cycle benchmarking has long faced methodological issues
as per what metrics to compare, how to compare them, and at what point in the
fuel cycle they should be compared
\cite{wilson2011comparing,guerin2009benchmark,piet2011assessment}.
The benchmarking mechanism described
here couples Gaussian process models (GP) \cite{rasmussen2006gaussian} to
dynamic time warping (DTW) \cite{muller}. Together these address how to
generate figures-of-merit (FOM) for common nuclear fuel cycle benchmarking
tasks.

Confusion in this area is partly because such activities
are not in fact `benchmarking' in the strictest validation sense. Most fuel
cycle benchmarks are more correctly called code-to-code comparisons, or
inter-code comparisons, as they compare simulator results. Importantly,
there is an absence of true experimental data for a benchmark or simulator
to validate against. Moreover, the number of
real world, industrial scale nuclear fuel cycles that have historically been
deployed is not sufficient for statistical accuracy. Even the canonical
once-through fuel cycle scenario has only been deployed a handful of times.
For other advanced fuel cycles, industrial scale data is even more scarce.
Since fuel cycle simulation is thus effectively impossible to validate,
non-judgmental methods of benchmarking must be considered.
Define a \emph{non-judgmental method} as one such that the
results of any given simulator is evaluated in reference to how
it performs against other simulators in such a way that acknowledges that
any and all simulators may demonstrate incorrect behavior. No simulator
by fiat produces the `true' or reference answer because all simulators
cannot be validated. To pick a reference solution would bias further analysis
towards that solution. This would, implicitly or explicitly, state that the
chosen simulator is the most valid simulator for purposes of the benchmark.
However, the aim of a non-judgmental benchmark is to avoid all such biases.

The other major conceptual issue with fuel cycle benchmarking is that there
is no agreed upon mechanism for establishing a figure-of-merit
that is uniform across all fuel cycle metrics of interest. For example,
repository heat load may be examined only at the end of the of the simulation,
separated plutonium may be used as a FOM wherever it peaks, and natural uranium
mined might be of concern only after 100 years from the start of the
simulation if resource scarcity is anticipated. Comparing at a specific point
in time fails to take into account the behavior of that metric over time and
can skew decision making. Additionally, the appropriate
time of comparison depends on the metric itself, and is thus not necessarily
the same point in time as for another metric. This is a necessary
side effect of picking a single time at which to compute metrics.
Furthermore, such FOMs are not useful for indicating why simulations differ,
only that they do. Moreover, if such FOMs match, this does not indicate
that the simulators actually agree; their behavior
could be radically different at every other simulation time and converge
only where the metric is evaluated.  The one caveat
here is that
equilibrium and quasi-static fuel cycle simulators are sometimes able to
ignore these issues because all points in time are treated equally.

Some dynamic FOMs do exist, such as some metrics seen in the recent
``Nuclear Fuel Cycle Evaluation and Screening - Final Report''
\cite{wigeland2014nuclear}. However, these typically require that the metric
data be too well-behaved for generic comparison purposes. Consider
total power produced [GWe] as the metric and a FOM for this metric as
the sum over time of the relative error
between the total power of a single simulator and the root-mean squared (RMS)
total power of all the simulators together. This RMS FOM has been suggested as
a candidate for the upcoming ``Open Fuel Cycle Benchmark''
\cite{mouginot2015ofcb}. However, such an FOM cannot be calculated
if the total power
time series have different lengths. Such differences could arise because
of different time steps (1 month versus 1 year) or because of different
simulation durations. Furthermore, suppose that a benchmark is posed as
"until transition" in a transition scenario. It would defeat the purpose of
the benchmark to force different simulators to have the same
time-to-transition if they nominally would calculate distinct transition
times. Therefore, a robust FOM should not impose the constraint of a uniform time grid.

The mechanisms used for benchmarking that have been discussed so far typically
incorporate neither uncertainty in the input parameters nor
modeling uncertainty from the simulator.
This is likely because most simulators do not compute uncertainty from their
inputs nor do they report uncertainties associated with the algorithms that
they implement.
Instead they rely on perturbation studies or stochastic wrappers around
the simulator. Furthermore, many metrics may rely on external nuclear
data (such as half-lives and cross-sections), which have corresponding
uncertainties that should effect benchmarking comparisons.
However, even if such error bars were available for
every point in a time series metric, the traditional benchmark FOM
calculations would ignore them. Integration over time of the metrics found in
\cite{wigeland2014nuclear} does not provide an uncertainty for the result,
and neither does the RMS FOM
\cite{mouginot2015ofcb}. Only the nominal value is used in these situations.

The purpose of this paper is to describe a methodology that generates FOMs
for all potential fuel cycle metrics in the same way such that the FOMs
incorporate any provided uncertainty.
Before going further, it is important to note that
most fuel cycle
metrics are time series and can be derived from the mass balances (or
mass flows) \cite{wilson2011comparing,guerin2009benchmark,piet2011assessment,wigeland2014nuclear}.
Additionally, many metrics have an associated total metric that can be
computed from the linear combination of all of its constituent features.
For example, total mass flows per time step are the sum of the mass flow
of each nuclide per time step
and total power generated is the sum of the power from each reactor type,
such as light water reactors (LWR) and fast reactors (FR). These attributes
are common to the overwhelming majority of fuel cycle metrics and so FOMs
may take advantage of such structures.

Gaussian process models are proposed here as a method to incorporate
metric uncertainties and make the analysis non-judgmental with respect to
any particular
simulator. Roughly speaking, a Gaussian process regression is a
statistical technique
that models the relationship between independent and dependent parameters
by fitting the covariance to a nominal functional form, or kernel.
The kernel may have as many or as few fit parameters (also called
\emph{hyperparameters}) as desired. One often used kernel is the squared
exponential, or Gaussian distribution \cite{rasmussen2006gaussian,hodlr}.
Linear kernels and Gaussian noise kernels are also frequently used
alternatives (see \cite{rasmussen2006gaussian,hodlr} again).

Using a Gaussian process is desirable because the model can be generated
from as many simulators as available and it will treat the results of
each simulator in the same manner. Unlike a relative error analysis, no
simulator needs to be taken as the fiducial case; the Gaussian process model
itself becomes the target to compare against.
Moreover, the covariances do not need to be known to perform the benchmark
since they are estimated by the Gaussian process. Furthermore, once the
hyperparameters are known, this can be used as a representative model for
any desired time grid. Additionally, the incorporation of the uncertainties
in a Gaussian process are known to be more accurate (closer means) than
assuming uncorrelated uncertainties \cite{rasmussen1996evaluation,ko2009gp}.
The trade off is that the model is less precise (higher standard deviations)
than the uncorrelated case \cite{hodlr}. Such a trade off is likely
desirable because no simulator
is necessarily more valid than any other simulator. For example,
in an inter-code comparison, an outlier simulator may be the
closest to the true solution, perhaps because it is higher fidelity than
the others. Thus it becomes important to quantify outliers rather than
discard them.

However, a Gaussian process model of a metric for a set of simulators
does not directly present itself as a FOM for that metric. Time series from
all simulators go into a Gaussian process and a representative model time
series comes out. The Gaussian process handles making the benchmark
non-judgmental and incorporating uncertainties. However, Gaussian processes
do not
perform a reduction in dimensionality from time series into to a simple
scalar. The dynamic time warping
technique is proposed as a method for deriving FOMs from the models that
does reduce the dimensionality to an easily comparable single number.

Dynamic time warping computes the minimal distance and path that it would
take to convert one time series curve into another. This procedure is highly
leveraged in audio processing systems, and especially in speech recognition
\cite{myers1980performance,muda2010voice}.
This is due to the DTW advantage that the two time series that are being
compared need not have the same time grid. In other words, it does not matter
if one time series is longer than
the other or if they have different sampling frequencies. The DTW distance
instead compares the shape of the curves themselves. This is beneficial as
a FOM derived from the DTW distance can thus represent the structural
similarity of the underlying time series.


There is nothing about the dynamic time warping algorithm that is specific
to speech recognition and the method
may be used for any time series analysis. For nuclear fuel cycle benchmarking,
the DTW distance can be used to compare the metric from each simulator to its
Gaussian process model. Using this as an FOM has the advantage of
incorporating information from the whole time series, rather than just a
specific point in the cycle.

Many benchmarking studies also wish to create a rank ordering of parameter
importance over all simulators. Examples of such benchmark questions include,
``In a transition scenario,
which reactor contributes the most to the total generated power?'' and
``Which
nuclides are most important to the repository heat load?'' DTW distances
of Gaussian process models of the constient parameters (e.g. the power from
each reactor type) to a Gaussian process model of the total (e.g. total
generated power) can be used as a FOM itself or to derive other FOMs.
This paper proposes a novel contribution metric
that is taken to be a normalized version of the DTW
distance for such rank ordering activities.

Additionally note that a fuel cycle benchmark is distinct from a
fuel cycle evaluation, such as the evaluation presented in
\cite{wigeland2014nuclear}.
In a benchmark, a single scenario (such the once-through or a transition
scenario) is analyzed with respect to the results coming from multiple
simulators. In an evaluation, on the other hand, the focus of the analysis
is on the performance differences between many scenarios. If many simulators
are used for each scenario in an evaluation, an inter-code comparison may
be a step within the evaluation. The methodology described in this paper
applies to such a benchmarking step. However,
the method here requires modification to enable comparisons
between different scenarios, which is the core feature of a fuel cycle evaluation.
Such additions to the methodology
are not present in this paper, whose focus is strictly benchmarking, but will
be forthcoming in future work that builds upon the method presented here.

The remainder of this paper takes a narrative approach that walks through
a fictitious example benchmarking activity. Example generated power data
(in gigawatts electric [GWe])
from DYMOND \cite{yacout2005modeling,feng2015dymond} and Cyclus
\cite{DBLP:journals/corr/HuffGCFMOSSW15,cyclus_v1_0} are used.
The underlying fuel cycle being modeled is an
LWR to FR transition scenario that occurs over 200 years, starting from 2010.
This data should be regarded as for demonstration purposes only. No deep
fuel cycle truths should be directly inferred and the data itself should be
considered preliminary. However,
using results from actual fuel cycle simulators shows how
non-judgmental benchmarking works in practice. Arbitrarily constructed
or randomly generated data could
have been used instead, but this demonstration of the method is more
motivating to future benchmarking activities.

In \S \ref{setup}, the benchmark problem is set up, mathematical notation is
introduced,
and the raw data from the simulators are presented. In \S \ref{gp}, Gaussian
process
modeling is introduced. In \S \ref{dtw}, the dynamic time warping concept is
discussed. In \S \ref{contribution}, the novel contribution metric is
derived.
\S \ref{filtering} warns against standard time series filtering
techniques. And finally, \S \ref{conclusion} contains concluding remarks
and ideas for future work in fuel cycle benchmarking and for
extending the mechanisms presented here.

\section{Benchmarking Setup}
\label{setup}

Suppose that there are $S$ simulators, indexed by $s$. The exact order
of these simulators does not matter, however a consistent ordering should
be used. In the demonstration here, there are two simulators with $s=0$
being DYMOND and $s=1$ for Cyclus.

Furthermore, for any feature or metric there may be $I$ partitions,
indexed by $i$. In this example benchmark, the total generated power metric
is studied and has constituent components of the power generated by LWRs
($i=0$) and
the power generated by FRs ($i=1$).  Again, the ordering of these is not
important, only that the ordering is consistent. An alternative example
that will not be examined here is the mass flow, which may be partitioned
by its individual nuclides or by chemical element.

Now, denote a metric as a function of time $t$ for a given simulator and
component as $m_s^i(t)$. For many metrics of interest
to nuclear fuel cycle analysis, the following equality holds:
\begin{equation}
m_s(t) = \sum_i^I m_s^i(t)
\end{equation}
Thus $m_s(t)$ is the total metric over all constituent parts for a given
simulator. This linear combination is useful when calculating contributions
(as seen in \S\ref{contribution}) but is not needed to compare various
simulators to a Gaussian process model (see \S\ref{dtw}).

Additionally, call $u_s^i(t)$ the uncertainty of the metric
$m_s^i(t)$. Note that $u$ is also a time series for each simulation for
each component. If uncertainties are not known, this can be set to floating
point precision (which states that the metric is as precise as possible) or
some nominal fraction of the value (10\%, 20\%, etc.). It is, of course,
much preferred for the simulator to compute
uncertainties directly. However, this is often not supported in the underlying
simulator. Choosing a nominal uncertainty, even if it is floating point
precision, must suffice in such cases.

\begin{figure}[htb]
\centering
\includegraphics[width=0.45\textwidth]{gwe-dymond.eps}
\includegraphics[width=0.45\textwidth]{gwe-cyclus.eps}
\caption{The generated power in [GWe] as a function of time for the DYMOND and
Cyclus simulators for both LWRs and FRs.}
\label{gwe-simulators}
\end{figure}

The demonstration here will use the generated power from LWRs and FRs in
a transition that covers 200 years. The simulation should start with
90 GWe generated solely by LWRs and meet a 1\% annual growth in demand over the
lifetime of the simulation. Figure \ref{gwe-simulators} shows the component time
series curves for both DYMOND and Cyclus.

The difference between the LWR curves in Figure \ref{gwe-simulators} stem
partially from how the simulators choose to implement the initial conditions.
DYMOND uses an initial condition that
begins with a fleet that produces 90 GWe in 2010 and whose reactors are
decommissioned on a staggered schedule. The Cyclus simulation, on the other
hand, begins the simulation in 1960 and deploys two reactors per year for
the first fifty years. This yields 100 LWRs in 2010 that have a refueling
schedule and capacity factor such that 90 GWe of power is generated. Since
Cyclus is agent-based, these reactors then decommission themselves naturally
given their 60 year lifespan. New reactors are subsequently deployed after
2010.

The difference between the FR curves in Figure \ref{gwe-simulators} come
from modeling differences in the simulators themselves. DYMOND does not
restrict FRs from operating once they are deployed even if fresh fuel is not
immediately available. Cyclus, however, does model fuel shortages and
reactors will sit idle until valid fresh fuel becomes available. Furthermore,
in the Cyclus simulation here (i.e. not a limitation of Cyclus itself),
all reactors are deployed only in January of each the year.
This leads to the oscillations seen at later times (2135+) because all
FRs that begin on the same time step will shut down for their first
refueling at the same time, all of those that obtain fuel will go down for
their second refueling at the same time, and so on. As more FRs are deployed,
this effect becomes more pronounced. DYMOND, on the other hand, models a
scalar capacity factor and does not produce these oscillations.

The next section describes how to
construct Gaussian process models from the data set presented above.

\clearpage

\section{Gaussian Process Modeling}
\label{gp}
The study of Gaussian processes is rich and deep and more completely covered by
other resources, such as \cite{rasmussen2006gaussian}. Here only the barest of
introductions to this topic are given as motivated by the regression problem at
hand. For nuclear fuel cycle benchmark analysis, Gaussian processes will be used
to form a model of the metric time series over all $S$ simulators.

A Gaussian process is defined by mean and covariance functions.
In general, these functions may be parameterized by many independent and
many dependent variables.  However, for purpose of this paper, a single
independent parameter, time, is sufficient to demonstrate the method without
loss of generality.
Here, the mean function $\mu(t)$ is taken to be the expectation value $\E$ of
the input functions, which here are the time series for all simulators. The
covariance function $k(t, t^\prime)$ is the expected value of the input
functions to the mean. Symbolically,
\begin{equation}
\label{mean-func}
\mu(t) = \E\left[m_s^i(t)\right] = \E\left[m_0^i(t), m_1^i(t), \ldots\right]
\end{equation}
\begin{equation}
\label{covar-func}
k(t, t^\prime) = \E\left[(m_s^i(t) - \mu(t))(m_s^i(t^\prime) - \mu(t^\prime))\right]
\end{equation}
The Gaussian process $\GP$ thus approximates the metric given information
from all simulators. This is denoted either using functional or operator notation as follows:
\begin{equation}
\label{gp-def-approx}
m_*^i(t) \approx \GP\left(\mu(t), k(t, t^\prime)\right) \equiv \GP m_s^i
\end{equation}
Let $*$ indicate that the variable is associated with
the model, rather than any of the raw simulator data.


For Gaussian process regression, a functional form for the covariance
$k(t, t^\prime)$ needs to be provided.
This is sometimes called the kernel function and contains the free parameters
for the regression, also called hyperparameters. How the hyperparameters are
defined is
tied to the definition of the kernel function itself. The values for the
hyperparameters are determined
via optimization of the maximal likelihood of the value of the metric function.
While there are many possible kernels, a standard and generically applicable one
is the exponential squared kernel, as seen in Equation \ref{exp2-kernel}:
\begin{equation}
\label{exp2-kernel}
k(t, t^\prime) = \sigma^2 \exp\left[-\frac{1}{2\ell}(t - t^\prime)^2 \right]
\end{equation}
Here, the length scale $\ell$ and the signal variance $\sigma^2$ are the
hyperparameters.

Now, define a matrix $\K$ such that the element at the $t$-th row and $t^\prime$-th
column is given by Equation \ref{exp2-kernel}. If a vector of training metric
values $\mathbf{m}$ is defined by concatenating metrics $m_s^i(t)$ for all
simulators
for all times $T$, then the logarithm of the likelihood of the obtaining
$\mathbf{m}$ is seen to be:
\begin{equation}
\label{log-p}
\log p(\mathbf{m}|T) = -\frac{1}{2}\mathbf{m}^\top\left(\K + u^2\I\right)^{-1}\mathbf{m}
                       -\frac{1}{2}\log\left|\K + u^2\I\right|
                       -\frac{n}{2}\log 2\pi
\end{equation}
Here, $u$ is the modeling uncertainty as denoted in the previous section,
$\I$ is the identity matrix, and $n$ is the number of training points (the
sum of the lengths of all of the time series). The hyperparameters $\ell$ and
$\sigma^2$ may be varied such that Equation \ref{log-p} is minimized.
In this way, an optimal model for the metric is obtained.

Finally, suppose we want to evaluate the Gaussian process regression at a
series of time points $\mathbf{t_*}$.
The covariance vector between this time grid and the training data is denoted
as $\mathbf{k}_* = \mathbf{k}(\mathbf{t_*})$. The value of Gaussian process
model of the metric function on the $\mathbf{t_*}$ time grid is thus seen to be:
\begin{equation}
\label{metric-model}
\mathbf{m}_*(\mathbf{t}_*) = \mathbf{k}_*^\top \left(\K + u^2\I\right)^{-1}\mathbf{m}
\end{equation}
A full derivation of Equations \ref{mean-func}-\ref{metric-model} can be found in
\cite{rasmussen2006gaussian}. This resource also contains detailed discussions of
how to optimize the hyperparameters, efficiently invert the covariance
matrix, and compute the model values. Often the model will be evaluated at
a point on the original time grids. However, the model may also be used in an
interpolative fashion for points that do not lie on either of the original
time series but still within the overall time domain. The model is also capable
of being used in a predictive fashion to extrapolate out to compute points in
the future or past of all time series that comprise the model.

In practice, though, a number of free and open source implementations of Gaussian
process regression are readily available. In the Scientific Python ecosystem, both
scikit-learn v0.17 \cite{scikit-learn} and George v0.2.1 \cite{hodlr} implement
Gaussian process regression. For the remainder of this paper, George is used
for its superior performance characteristics and easier-to-use interface.

For purposes of nuclear fuel cycle benchmarking, Gaussian processes can be used to
create models of each component feature (e.g. LWRs and FRs) from the
results of all simulators together. Other regression techniques could also be used
to create similar models.  However, Gaussian processes have the advantage of
incorporating modeling uncertainty, as seen above. The optimization of the
hyperparameters yields the most accurate results for the model. Furthermore, the
choice of kernel function can be tailored to the functional form of the metric, if
necessary. The exponential squared function was chosen because it is generic. If the
metric is periodic, for instance, a cosine kernel may yield a more precise model.
So while other regression techniques could be used, Gaussian processes
encapsulate the correct behavior necessary for a non-judgmental benchmark while
also remaining flexible to the particularities of the metric under examination.

Using the sample data from \S\ref{setup}, three models can be constructed:
generated power from LWRs $m_*^{\LWR}(t)$, generated power from FRs $m_*^{\FR}(t)$,
and total generated power $m_*(t)$. Assuming only floating point precision as
the metric uncertainty, these Gaussian process models can be seen in Figures
\ref{gwe-model-lwr} - \ref{gwe-model-total} along with the original time series
from the simulators that train the model.
It is important to note that even though the sample data is only for two simulators,
this method can handle any number of simulators, each with their own time grid,
without modification.

\begin{figure}[htb]
\centering
\includegraphics[width=0.9\textwidth]{gwe-model-lwr.eps}
\caption{The Gaussian process model of the generated power from LWRs
as a function of time as well as the results from the simulator that served as a
training set for the model.}
\label{gwe-model-lwr}
\end{figure}

\begin{figure}[htb]
\centering
\includegraphics[width=0.9\textwidth]{gwe-model-fr.eps}
\caption{The Gaussian process model of the generated power from FRs
as a function of time as well as the results from the simulator that served as a
training set for the model.}
\label{gwe-model-fr}
\end{figure}

\begin{figure}[htb]
\centering
\includegraphics[width=0.9\textwidth]{gwe-model-total.eps}
\caption{The Gaussian process model of the total generated power
as a function of time as well as the results from the simulator that served as a
training set for the model.}
\label{gwe-model-total}
\end{figure}

\clearpage

Also note that Figure \ref{gwe-model-total} displays the model of the total
generated power and not the total of the constituent models. Symbolically,
\begin{equation}
\label{total-model}
m_* \approx \GP \left[\sum_i^I m_s^i\right] \ne \sum_i^I \GP m_s^i
\end{equation}
This is because the uncertainties are applied differently in these two cases.
Moreover, the hyperparameter
optimization would not be consistent if one were to sum up constituent
models. It is thus considered safer
to sum over the features for each simulator individually before applying the
regression.

Thus far the metric data has had effectively zero uncertainty, but one of the
desirable features of the Gaussian process regression is that it may account
for uncertainties. As a thought experiment, suppose that the time
series data happened to be much sparser and that
the uncertainty associated with each value started off at zero and then grew at
a rate of 1\% per decade. A model of the total generated power with this uncertainty
is shown in Figure \ref{gwe-model-total-with-uncertainty}.

\begin{figure}[htb]
\centering
\includegraphics[width=0.9\textwidth]{gwe-model-total-with-uncertainty.eps}
\caption{The Gaussian process model of the total generated power
as a function of time. The training set data is relatively sparse and its uncertainty
has a 1\% per decade growth rate. The model curve is evaluated at every year. The
gray envelope represents two standard deviations from the model, as computed by
the Gaussian process.}
\label{gwe-model-total-with-uncertainty}
\end{figure}

In Figure \ref{gwe-model-total-with-uncertainty}, note that as the model moves
farther in time from the training data the standard deviation grows. Furthermore,
as the uncertainty in the training data grows, the model itself degrades. Artifacts
from the choice of kernel begin to be visible in model whenever the uncertainties are
relatively high. Both of these match intuition about how uncertain systems should
work.

Again, even though these uncertainty features are highly desirable in
generic benchmarking applications, most nuclear fuel
cycle simulators do not report uncertainties along with the metrics they compute.
For this reason, the remainder of the examples in the paper will use the
zero-uncertainty models as presented in Figures \ref{gwe-model-lwr} -
\ref{gwe-model-total}.

\clearpage
\section{Dynamic Time Warping}
\label{dtw}

Now that there are representative models of all time series, the issue at hand is
how to compare these models. Dynamic time warping
is a method for computing the distance between any two time series. The time
series need not be of the same length.  Furthermore, the time series may be
decomposed into a set of representative spectra and the DTW may still be applied.
The distance computed by DTW is a measure of the changes that would need to be
made to one time series to turn it into (warp) the other time series. Thus,
the DTW distance is a measure over the whole time series, and not just a
single characteristic point. As with Gaussian processes, a number of more thorough
resources, such as \cite{muller}, cover dynamic time warping in greater detail.
Here a minimal introduction is given that enables meaningful figure-of-merit
calculations.

With respect to nuclear fuel cycle benchmarks, there are two main DTW applications.
The first is to compute the distance between a Gaussian process model and each of
the simulators that made up the training set for that model. This gives a
quantitative measure
of how far each simulator is from the model and can help determine which
simulators are outliers. To maintain a non-judgmental benchmark, though, it is
critical to not then use this information to discard outliers.  Rather, outlier
identification should be used as part of an inter-code comparison. If one simulator
is an outlier for a given metric, the reasons for this should be investigated.
For example, the outlier simulator may be at a higher fidelity level, there may be
a bug in the outlier, or there may be a bug in all other simulators. Identifying
outliers for many metrics could help discover the underlying cause of any
discrepancies.

The second application of dynamic time warping to benchmarking is to compare
the constituent
feature models to the total model.  Distances computed in this manner allow
for a rank ordering of the components.  This enables the benchmark to make claims
about which features drive the fuel cycle metric most strongly over the whole
simulation time domain and for all simulators. Traditionally, the simulators have
to agree
within nominal error bounds ($<5\%$) for a benchmark to make such a claim.  Here,
the simulators need not necessarily agree since the Gaussian process models are
used as representatives.  In this application, it is useful to recast the DTW
distance as a measure of contribution.  Contribution FOMs will be presented in
\S\ref{contribution}.

For any two time series, dynamic time warping consists of three mathematical objects:
the distance $d$, a cost matrix $C$, and a warp path $w$. The cost matrix
specifies how far a point on the first time series is from another point on the
other time series.  The warp path is then the minimal cost curve through this
matrix from the fist point in time to the last. The distance, therefore, is the
total cost of traversing the warp path.

The first step in a dynamic time warping algorithm is to compute the cost matrix.
Suppose that the first
time series $x$ has length $A$ indexed by $a$ and the second time series $y$ has
length $B$ indexed by $b$. It is helpful to define an $A\times B$ matrix $\Delta L$
that is the $L_1$ norm of the difference of time series $x$ and $y$:
\begin{equation}
\label{delta-l1}
\Delta L_{a,b} = \left|x_a - y_b\right|_1
\end{equation}
The cost matrix $C$ is then an $A\times B$ sized matrix that is defined by the
following recursion relations:
\begin{equation}
\label{cost-matrix}
\begin{split}
C_{1,1} & = \Delta L_{1,1}\\
C_{1,b+1} & = \Delta L_{1,b} + C_{1,b}\\
C_{a+1,1} & = \Delta L_{a,1} + C_{a,1}\\
C_{a+1,b+1} & = \Delta L_{a,b} + \min\left[C_{a,b}, C_{a+1,b}, C_{a,b+1}\right]
\end{split}
\end{equation}
The boundary conditions in Equation \ref{cost-matrix} are equivalent
to applying an infinite cost to any $a$ or $b$ less than or equal to zero.
The units of the cost matrix are the same as the units of the metric. However, the
scale of the cost matrix is geometrically larger than the metric itself. This is
due to the compounding nature of the recursive definition of $C$.

Given $C$, the warp path is thus a sequence of coordinate points that can then be
computed by traversing backwards through the matrix from $(A, B)$ to $(1, 1)$.
For a point $w_p$ in the warp path, the previous point $w_{p-1}$ is given by
where the cost is minimized among the locations one column over $(a,b-1)$,
one row over $(a-1,b)$, and one previous diagonal element to $(a-1,b-1)$.
Symbolically,
\begin{equation}
\label{warp-path}
w_{p-1} = \argmin\left[C_{a-1,b-1}, C_{a-1,b}, C_{a,b-1}\right]
\end{equation}
The maximum possible length of $w$ is thus $A + B$ and the minimum length is
$\sqrt{A^2 + B^2}$. The warp path itself could potentially serve as a FOM.
However, doing so would not take into account the cost along this path.

Therefore, the distance $d$ is defined as a FOM which does include for the cost of the
warp.  Due to the recursion relations used to define the cost matrix, $d$ can be
stated succinctly as:
\begin{equation}
\label{d-calc}
d = \frac{C_{A,B}}{A + B}
\end{equation}
That is, $d$ is the final value of the cost matrix divided by the maximal length
of the warp path. When $d$ is zero, this indicates that the two time series
are exactly the same. Thus small, positive values of $d$ suggest that the
time series are similar to one another while larger values denote that they
are distinct. The magnitude of large and small is dependent on the range of
the time series. A simple definition of large is the maximum
value of either of the time series multiplied by the length of the time series,
while small is a representative fraction of this (e.g. 1\%, 5\%, or similar).

For all practical purposes in nuclear fuel cycle benchmarking, $A$ and $B$ can be
forced to have the same
value. This is because the Gaussian process model can be used to predict a time series
with whatever time grid is desired.  The advantage of using a regression model
is that it allows the analyst to force the same time grid.  The advantage of
dynamic time warping is that ensuring the same time grid is not necessary.
Coupling Gaussian processes and DTW together is a more robust analysis tool
than the methods individually.

\begin{figure}[htb]
\centering
\includegraphics[width=0.9\textwidth]{cost-lwr-model-to-lwr-cyclus.eps}
\caption{Heat map of the cost matrix between the Gaussian process model
for LWRs, $m_*^\LWR(t)$, and the Cyclus LWR time series, $m_\CYCLUS^\LWR(t)$.
The warp path is superimposed as the white curve on top of the cost matrix.}
\label{cost-lwr-model-to-lwr-cyclus}
\end{figure}

Figure \ref{cost-lwr-model-to-lwr-cyclus} displays an example cost matrix
as a heat map for the DTW between the LWR Gaussian process model
$m_*^\LWR(t)$ and the original Cyclus LWR time series $m_\CYCLUS^\LWR(t)$.
Additionally, the warp path between these two is shown as the white curve
on top of the heat map. Note that while $w$ is monotonic along both time axes, the
path it takes minimizes the cost matrix at every step. Higher cost regions,
seen as the brighter and more yellow or green areas in Figure \ref{cost-lwr-model-to-lwr-cyclus},
have the effect of pushing the warp path along one axis or another. The
distance between these two curves $d(m_*^\LWR, m_\CYCLUS^\LWR)$ is computed
to be 1.053 GWe. To give a sense of scale to this distance, the maximum of
the cost matrix divided by the
maximum path length, or $\max[C]/(A+B)$, is 34.86 GWe.  This
indicates that that Cyclus LWR time series matches the model to within 3\%.

The process of dynamic time warping model generated data to raw simulator data can be
repeated for all combinations of simulators and features. The
$d(m_*^i, m_s^i)$ that are computed may then directly serve as a FOM for
outlier identification. Only distances between the same feature $i$ may be compared,
as they share the same model. For instance, it is valid to compare
$d(m_*^\LWR, m_\DYMOND^\LWR)$ and $d(m_*^\LWR, m_\CYCLUS^\LWR)$. However,
it is not valid to compare $d(m_*^\LWR, m_\DYMOND^\LWR)$ and
$d(m_*^\FR, m_\DYMOND^\FR)$.  In the situation, where the simulators
have different time grids, the model can be evaluated with each time grid
prior to computing the corresponding DTW distances and the comparison will
remain valid.

The interpretation of the DTW distances should be, ``How much effort does
it take to turn one time series into another.'' This effort is measured
in the same units as the original time series, but with a compounded
scale (due to the construction of the cost matix in Equation \ref{cost-matrix})
that is then reduced by
the maximum warp length (by Equation \ref{d-calc}). Thus if the fuel
cycle metric has physical units, such as GWe or kg, the distance will have
the same physical units. In a heuristic sense, the DTW distance represents
the minimum amount (in units of the metric) that would have to be transfered
between the time series for the two curves to be the same. However, it is
problematic to take this heuristic notion at face value because of the
scale differences between the original metric and the DTW distance.
For the metric and the $d$ values to be directly comparable, the DTW distance
would need to be decompounded.

\begin{table}[htb]
\centering
\caption{Distances [GWe] between models and simulators for all combinations of
simulators (DYMOND and Cyclus) and metric features (generated power for
LWRs, FRs, and in total). The simulators are presented in the rows and the
features are given as columns. Distances, as computed by
Equation \ref{d-calc}, may only be compared along each column.}
\label{d-compare}
\begin{tabular}{l||c||c||c|}
                & \textbf{LWR} & \textbf{FR} & \textbf{Total} \\
\hline
\textbf{DYMOND} & 1.452        & 2.783       & 3.022          \\
\hline
\textbf{Cyclus} & 1.053        & 3.732       & 3.984          \\
\hline
\end{tabular}
\end{table}

Table \ref{d-compare} gives the distances for all simulator and feature
combinations in the sample data presented in \S\ref{setup}. Only the differences between DYMOND and
Cyclus for the same feature may be directly compared.  However, Table \ref{d-compare}
does indicate that DYMOND is closer to the model since both the FR and total
generated power distances are closer for DYMOND than for Cyclus.  As many
simulators as desired could be added to the benchmark and distances could
be tallied for them as well. At a sufficient number of simulators, usual
statistics (mean, standard deviation) along each column may be computed.
Simulators whose metrics fall outside of a typical threshold of the mean
(one or two standard deviations) would then be considered outliers and
subject to further inter-code comparison. The two simulators here are
for demonstration purposes and neither can be said to be an outlier in a
non-judgmental way. Recall, though, that outliers determined in this way
may stem from the more correct simulator. The term outlier is not a
condemnation on its own.

Unless all simulators produce precisely the same time series, it is
expected that the model will differ, at least in part, from all simulators.
Thus the DTW distances are usually expected to be non-zero. How significant
the differences in these distances are can be determined from the mean
and standard deviations of these distances.  However, the data presented
in Table \ref{d-compare} is not sufficient to make any claims with respect
to such significance because only two simulators are available. A benchmark
study with many participating simulators would be capable of making more
concrete claims in this regard.

The second application of DTW to nuclear fuel cycle benchmarking is as a
FOM for contribution of constituent features to a total metric. For this application,
compute the distance between the model of the total and the model of the
part, namely $d(m_*^\Total, m_*^i)$ for all $i \in I$. This provides a
measure of how much the total metric is determined by a particular part
over the whole time series for all simulators.
Using this measure is only reasonable if the total metric is known
to be the sum of its constituents.  This assumption is valid for
a large percentage of nominal benchmarking metrics. In particular,
mass flows and generated power both follow this rule. Metrics that are
linear transformations of these basic metrics (such as decay heat or
radiotoxicity) will also conform to this constraint. The distances computed here
can then be used to rank order the importance of each constituent feature.
Smaller distances are closer to the total and thus have a greater impact.

In the sample data here, the total generated power model can be compared to
the models for LWR and FR generated power. The value of
$d(m_*^\Total, m_*^\LWR) = 97.010$ while $d(m_*^\Total, m_*^\FR) = 19.503$.
These two values represent a comparison of a partition of a total metric,
as described above.  Furthermore, since this is a comparison of models and not
of simulators, this comparison is non-judgmental with respect to the
simulators that comprise the models. Therefore, the any claims about the
impact of different constituents of the partition (LWRs vs FRs) is more general
than if the impact had been measured only with a single simulator.
As expected, because this is a 1\% growth scenario, the FRs are a larger
driver of the transition system (lower $d$ value) than the
LWRs (higher $d$ value) over the time domain examined. Alternatively, if this
were a decrease scenario (i.e. negative growth)
that retained transition, it is not clear \emph{a priori} whether LWRs or FRs would
provide a greater impact to the total power generated. The method detailed
here, though, grants a mechanism for comparing models in all potential
scenarios. Figures
\ref{cost-total-model-to-lwr-model} \& \ref{cost-total-model-to-fr-model}
show the cost matrices and warp paths for these two model-to-model cases.
Notice that in the
FR case, the warp path is effectively flat until the FRs are deployed in
significant numbers.

\begin{figure}[htb]
\centering
\includegraphics[width=0.9\textwidth]{cost-total-model-to-lwr-model.eps}
\caption{Heat map of the cost matrix between the Gaussian process model
for total generated power, $m_*^\Total(t)$, and the LWR model,
$m_*^\LWR(t)$.
The warp path is superimposed as the white curve on top of the cost matrix.}
\label{cost-total-model-to-lwr-model}
\end{figure}

\begin{figure}[htb]
\centering
\includegraphics[width=0.9\textwidth]{cost-total-model-to-fr-model.eps}
\caption{Heat map of the cost matrix between the Gaussian process model
for total generated power, $m_*^\Total(t)$, and the FR model,
$m_*^\FR(t)$.
The warp path is superimposed as the white curve on top of the cost matrix.}
\label{cost-total-model-to-fr-model}
\end{figure}

In summary, dynamic time warping yields a mechanism for benchmarks to
compare various simulators and models. As a tool for
outlier identification, it allows for each simulator to use its own native
time grid. As a tool for performing rank ordering of features, the DTW distances
are a perfectly functional FOM. However, lower distances implying higher
impact may run counter to intuition. The potential for confusion, therefore,
makes it less than ideal as a FOM on its own. A remedy for this is presented
in the next section in form of a new contribution FOM.

\clearpage
\section{Contribution}
\label{contribution}

In \S\ref{dtw}, the dynamic time warping distance was presented as an
FOM for measuring the similarity between time series, whether they
were model-to-simulator comparisons or model-to-model comparisons.
The distances alone, though, have the inverse meaning when trying to
compute a FOM for contribution to a fuel cycle.  For example, the
question may be posed as, ``Do LWRs or FRs impact the total power generated
more over the entire fuel cycle time domain?''
While distances can answer this question, smaller
disantances have higher impact. Preferably, a higher FOM score would yield a
higher impact. Furthermore, the distances in the previous section have
the same units as the cost matrix and are similarly bounded. A better
FOM for contribution would be unitless and defined on the range $[0,1]$.

Thus, let $c$ be the \emph{contribution} FOM that satisfies the above
constraints. To define $c$, first define $D$ as the maximal possible
distance from the model of the total. Recall that this time series has length $A$.
$D$ is, therefore, the L1 norm of the model of the total time series divided
by twice its length.
\begin{equation}
\label{D-def}
D(m_*^\Total) = \frac{\left|m_*^\Total\right|_1}{2A}
\end{equation}
This is the equivalent to computing the DTW distance
between $m_*^\Total$ and the curve where the metric is zero for all time
(i.e. the t-axis itself).  This relies on the notion that
metric is necessarily non-negative everywhere.  If the metric is allowed to
be negative, another baseline curve could be chosen. $D$ would then be
computed as the DTW distance between the total model and this baseline.
However, in most cases the metrics are not allowed to be negative,
a baseline of zero is suitable, and Equation \ref{D-def} applies.

The contribution figure-of-merit of a given partition to the total is thus
defined as follows:
\begin{equation}
\label{cont}
c^i = 1 - \frac{d(m_*^\Total, m_*^i)}{D(m_*^\Total)}
\end{equation}
$D$ is seen to normalize the model-to-model distance while subtracting this
ratio from unity makes higher contribution values more important.
Using the sample data, $c^\LWR = 0.298$ and $c^\FR = 0.859$. This again shows
that the FRs are more important to the total power of the whole cycle.
Here though, higher contribution scores yield higher importance and the values
are always between zero and one.

Note that even though $c$ is a fraction, it is not normalized across
partitions. Namely, the sum of all contributions for all $I$ partitions
is on the following range, which is not $[0, 1]$:
\begin{equation}
\label{sum-c-range}
0 \le \sum_i^I c^i \le I
\end{equation}
It is easy to imagine an alternative FOM that divides $c^i$ by $I$. However,
this was not done here because the choice of $I$ (the number of partitions)
can be made arbitrarily large.  In the sample calculations $I=2$ for LWRs and
FRs.  However, $I$ could have been set to 3 by including small modular reactor
(SMRs) which have not yet been built and produce no power.  $I$ could then be
increased to 4 or higher by including more non-existent reactor types.

Dividing the contribution by $I$ is not sufficient to
normalize the sum of the $c^i$, in general, because the
contribution is inherently a cumulative measure. If a component ever had a
non-zero value, it will always be seen to have contributed something.
Because of this constraint on the total, $\sum c^i$ can never
reach $I$ unless there is only a single component or the
total is zero valued everywhere (which implies that the constituents are also
zero). Thus, dividing $c^i$ by $I$ with the aim of normalizing the
sum of the $c^i$ is incorrect for any situation of interest to a fuel
cycle benchmark.

If a truly normalized version of contribution is desired, it must
use the sum of the actual contributions. Define the normalized contribution
as $|c^i|$,
\begin{equation}
\label{norm-ci}
\left|c^i\right| = \frac{c^i}{\sum_j^I c^j}
\end{equation}
The disadvantage with the normalized contribution is that all of the
individual component $c^i$ must be known prior to the normalization.
Furthermore, since the sum is typically greater than 1, the difference
between components is often smaller in the normalized form than in the
more pronounced unnormalized contribution.
The only advantages that $|c^i|$ confers over $c^i$ are that it is defined on
the range $[0,1]$ and that $\sum |c^i| = 1$. Otherwise, both versions of the FOM
have the same properties with respect to having a zero baseline, the
interpretation that higher values are higher impact to the total, and that they
are non-judgmentally derived from models rather than simulators.

\begin{figure}[htb]
\centering
\includegraphics[width=0.9\textwidth]{c-of-t.eps}
\caption{Contribution of LWRs and FRs to total generated power as a
function of time.}
\label{c-of-t}
\end{figure}

\begin{figure}[htb]
\centering
\includegraphics[width=0.9\textwidth]{normc-of-t.eps}
\caption{Normalized contribution of LWRs and FRs to total generated power
as a function of time.}
\label{normc-of-t}
\end{figure}

Both $c^i$ and $|c^i|$ can be viewed as a function of time.
Doing so in a nuclear fuel cycle benchmark may help identify artifacts in the
calculation that are the result of the time domain chosen for the benchmark itself.
For instance, the FR contribution is expected to overtake the LWR contribution.
If this does not occur or occurs very close to the maximum simulation time,
this implies that the time horizon of the benchmark should be increased.
Figures \ref{c-of-t} \& \ref{normc-of-t} display the contribution and
normalized contribution respectively for both LWRs and FRs over the full
time domain. From these figures, the point where FRs have a greater impact
to the total generated power
than LWRs is seen to be approximately year 2140.

It is worth noting that for all metrics for which the above contribution FOMs
are computable, direct integration of the Gaussian process models over
the whole time domain can also provide an analogous FOM. This is because the
contribution requirement that the metric have a linearly combined total
implies that the metric is also numerically integrable. A direct integration
could be substituted into Equation \ref{cont} to generate an FOM that also
has the property that higher values indicate higher impact
to the total. However, the contribution FOMs may still be desirable over
integration based FOMs.  This is because the DTW cost compounds the
differences in the time series. As a result, the DTW distance based
contributions are relatively more distinct than would be model integration
contributions. This extra emphasis on diverging contribution values may be
desirable to an inter-code comparison as the purpose of an FOM is to
highlight the differences.

\clearpage
\section{Cautionary Tale on Filtering}
\label{filtering}

It is tempting to insert standard filtering techniques from signal processing
after creating a Gaussian process model but prior to any dynamic time warping
calculations. The temptation comes from the fact that DTW is a signal
processing technique and so, hypothetically, using other well-known algorithms
before DTW would clean the signal and result in a smoother DTW comparison.
A fast Fourier transform (FFT) based low-pass filter
\cite{merletti1999standards,moreland2003fft} or
Mel-frequency cepstral coefficients (MFCC) \cite{muda2010voice,imai1983cepstral}
could potentially be used to reduce error in the model itself,
and thus make the contribution FOM more precise. Unfortunately, most
fuel cycle metrics are not well-formed candidates for such filtering strategies.
Including such filters as part of the analysis can easily lead to wildly unphysical
models.

Consider a simple low-pass filter where a 256 channel real-valued FFT frequency
transform is taken.  All but lowest 32 channels are discarded prior to the applying
the inverse transform. This algorithm removes the
high frequency jitter in the original signal, and
potentially allows for a better signal-to-noise ratio. Different values
for the number of total channels and the number of channels kept could be
used to create alternative low-pass filters. These values represent the
resolution of the transformation to and from the frequency domain. Keeping
only the coarsest eighth of the total number of channels should retain only
the general trends from the original time series.

\begin{figure}[htb]
\centering
\includegraphics[width=0.9\textwidth]{fft-lwr-model.eps}
\caption{Low-pass FFT filter of LWR Gaussian process model $m_*^\LWR$ alongside
the unfiltered model itself.}
\label{fft-lwr-model}
\end{figure}

Figure \ref{fft-lwr-model} shows the results of applying the low-pass filter
described above to the Gaussian process model of the LWR generated power,
$m_*^\LWR(t)$.  The filtered curve demonstrates at least three major problems.  The
first is that the values of the curve are allowed to be negative, which is
impossible for this (and many other) fuel cycle metrics.  The second is that
near the time boundaries ($t=2010$ and $t=2210$), the amplitude of the filtered model
is significantly higher than the unfiltered model. At $t=2210$, the metric
should be zero but
instead is 36.5 GWe. This is because of the oscillations in the filter that
begin around 2150 and grow in amplitude. These oscillations themselves
come from the fact that the FFT is attempting to model a flat line as a sum
of only 32 sinusoidal basis functions.
Thirdly, the shape of the curve itself is skewed to lower
times. The time at which the metric goes to zero should be near year 2150 but is
instead closer to year 2115.  All of these issues would severely distort any
DTW calculations that follow.

The reason behind these inconsistencies is that the FFT process is fundamentally
periodic.  However, using the annual time grid here, the LWR generated power metric
is not periodic. Neither is the modeling error for most fuel cycle metrics periodic
on an annual basis.
Thus, while well-intentioned, a low-pass filter is not generally applicable.

Alternatively, MFCCs provide a mechanism for converting a time series into a
set of power spectrum coefficient curves. In practice, many audio processing
libraries contain an out-of-the-box MFCC algorithm. For this paper,
the MFCC implementation provided by the librosa package v0.4.1
\cite{mcfee2015librosa} was used.
Since the dynamic time warping procedure
uses an $L_1$ norm to form the cost matrix, the MFCCs of two signals can be directly
compared. Each coefficient should roughly correspond in shape and amplitude to some
feature in the original signal.  Noisy, high frequency coefficients tend to be
very similar and so their contribution to a DTW distance is correspondingly less
than the contribution for lower mode coefficients. Coupling MFCC to DTW is an
extremely common method employed in speech recognition systems
\cite{muda2010voice,milner2002speech,sato2007emotion}.

\begin{figure}[htb]
\centering
\includegraphics[width=0.9\textwidth]{mfcc-lwr-model.eps}
\caption{Representative Mel-frequency cepstral coefficients (solid lines) of
an LWR Gaussian process model $m_*^\LWR$ alongside the model itself (dashed
line). The first seven coefficients are labeled in order of decreasing
significance.  The remaining, least-significant coefficients are visible
near the time axis.}
\label{mfcc-lwr-model}
\end{figure}

Figure \ref{mfcc-lwr-model} displays the MFCC curves of the LWR Gaussian
process model as well as the model itself. None of these curves, not even the
major coefficient (MFCC 0), resembles the actual model.
As with the low-pass filter, the MFCCs also have
uncharacteristic negative components.  Moreover,
the metric data is not sampled frequently enough to have meaningful
time windows. For the fuel cycle metrics here, there is only one data point per year
and the signal itself may change in a meaningful way each year. By comparison,
in speech recognition, audio is sampled on the order of 22050 Hz
\cite{EBUTECH3285,juang1991hidden}.
The data volume for fuel cycle benchmark metrics
is simply too low for MFCC transformations to capture the desired features.

Proceeding anyways, suppose the contribution measure is computed for the MFCCs
of LWR, FR, and
total generated power models.  In this case the LWR contribution is found to be
0.572 while the FR contribution is 0.899. Using the models directly as was done
in \S\ref{contribution}, the contribution values were 0.298 and 0.859 respectively.
This implies that using the MFCCs had the opposite effect as desired.  The MFCCs
added error to the FOM and made the LWR and FR contributions seem more alike
than the analysis without such filtering indicated.

Therefore filtering the models prior to dynamic time warping is a dubious practice
in the general case. In all likelihood, the metric does not meet the underlying
assumptions of the filter. The metric may not be periodic or may not be sampled
frequently enough. Sometimes it may be possible to construct a metric that does
meet these qualifications. For instance, the generated power could be sampled monthly
such that seasonal demand behavior is noticeable. Even in such a case, it is instead recommended
to pick a different kernel for the Gaussian process model such that these
periodic behaviors are captured.  The regression itself then takes on the role of
minimizing model uncertainty. Further filtering to this end becomes redundant and
dangerous.  Additionally, it is unlikely that
the majority of the simulators would be able to calculate such a high-fidelity metric.
That alone should disqualify such metrics or FOMs from any benchmarking study or
inter-code comparison.

\section{Conclusions \& Future Work}
\label{conclusion}

This paper demonstrates a robust method for generating figures-of-merit
for nuclear fuel cycle benchmarking activities by coupling Gaussian process
regression to dynamic time warping. This method takes advantage of modeling
uncertainties in fuel cycle metrics if they are known. It is also capable
of handling the situation where different simulators output metric data on
vastly different time grids. The distance computed by the dynamic time
warping can itself serve as the figure-of-merit. Additionally, the
distance can also be used to derive contribution and normalized contribution
figures-of-merit. These measures are valuable for non-judgmentally
determining the impact of different constituent measures to a total
metric, e.g. LWR versus FR power generation. The contribution metrics also
scale such that higher values imply higher impact. This is sometimes
more intuitive as compared to the DTW distance itself, where lower values
imply more similarity in the curves.

Any regression method could have been used to form a model. Similarly, any
mechanism for comparing two time series could have been used as a measure
of distance.  However, Gaussian processes and DTW were chosen because
fuel cycle simulator, and thus benchmarks and inter-code comparisons, lack
the ability to be experimentally validated.
It is not possible to fully specify a fuel cycle scenario,
physically construct the initial conditions by building actual facilities,
and then record how the cycle performs at a future date.
Going through this process for many scenarios and initial conditions that
a benchmark may examine is likewise impossible in practice.
Furthermore, using
historical data for validation provides too few cases for comparison and
each simulator could simply be tuned to precisely match historical events.
Thus, each simulator in a benchmark could be valid or they all could be
invalid. It is therefore necessary for the FOM to not skew for or against
any particular simulator. Gaussian process models as used here do not
judge the simulators differently. The DTW then takes into account the
cumulative effect of the whole time domain and does not preferentially
select certain times.

The sample benchmark presented here was very simple and was used for motivation
purposes only. It consisted of just
two simulators (DYMOND and Cyclus) and one metric (generated power) with
two components (LWR and FR).  However, both Gaussian processes and DTW
are inherently multivariate. In this paper, a univariate formulation of
Gaussian processes was all that was needed to demonstrate the method.
Likewise,
the $L_1$ norm of DTW can operate on vectors of time series as was seen
in the discussion of MFCCs. Therefore, higher dimensional forms of analysis could
be performed without modifying the fundamental method. For example, the Gaussian process could jointly model the
effect from many inputs onto the metric. Consider a benchmark which is formulated
to look at the generated power as a function of time and the power demand curve
jointly.
In this case, a two dimensional GP model would be used. Alternatively,
suppose that a matrix time series of the all individual nuclide mass flows
are available. DTW is still able compute the distance between two
such matrices. This would yield a measure of how the mass flows themselves
differ - taking into account each nuclide component - without relying on a
collapsed one dimensional total mass flow curve.
This is valuable as a FOM because it provides a measure of the whole fuel
cycle as a function of each facility, each nuclide, and time
without needing to explicitly examine the individual FOMs for each facility
and each nuclide.
Such cases will be considered in
future work as real inter-code comparison data becomes available.

Furthermore, this work focused only on benchmarking and inter-code comparison
activities.
However, the FOM calculations presented here could also be used to evaluate
different fuel cycle scenarios. DTW distances could be computed between
a typical once-through scenario and an LWR-to-FR transition
scenario, or any other proposed scenario. This provides a measure for
comparing the relative cost (in units of the metric, not necessarily
economic) for selecting one cycle over another.
Such measures have the advantage of being unbiased with respect to the
simulators that comprise the evaluation study; no reference or best-performing
simulator needs to be chosen (as was done in \cite{wigeland2014nuclear}).
Additionally, because DTW may handle different length time series,
transition scenarios that complete transition at different times can be
compared directly without the need to run all scenarios out to the time of
the longest transition.
Furthermore for FOMs that seek to encompass all fuel cycle facilities,
the distance measures may need to employ zero-valued `ghost'
facilities to establish a common basis between scenarios.  For example,
imaginary FRs with zero material flowing through them may need to be added
to a once-through scenario to be able to compute a DTW mass balance distance
to an LWR-to-FR transition scenario. For fuel cycle evaluation activities,
it is expected that the union of all types of facilities across all scenarios
will be required.
The work here, thus,
should be seen as a stepping stone to further fuel cycle scenario evaluation
efforts. It is this upcoming evaluation application is anticipated to more
directly benefit decision makers with respect to studies such as
\cite{wigeland2014nuclear} than the benchmark application presented here.

\section*{Acknowledgements}
\label{acknow}
The author would like to express deep gratitude to Dr. Bo Feng of
Argonne National Lab for providing the DYMOND data used throughout this
paper.

%%%%%%%%%%%%%%%%%%%%%%%%%%%%%%%%%%%%%%%%%%%%%%%%%%%%%%%%%%%%%%%%%%%%%%%%%%%%%%%%
\bibliographystyle{ans}
\begin{thebibliography}{10}

\bibitem{wilson2011comparing}
P.~P. WILSON,
\newblock ``Comparing nuclear fuel cycle options,''
\newblock {\em Observation and Challenges} (2011).

\bibitem{guerin2009benchmark}
L.~GU{\'E}RIN et~al.,
\newblock ``A benchmark study of computer codes for system analysis of the
  nuclear fuel cycle,''
\newblock , Massachusetts Institute of Technology. Center for Advanced Nuclear
  Energy Systems. Nuclear Fuel Cycle Program (2009).

\bibitem{piet2011assessment}
S.~J. PIET and N.~R. SOELBERG,
\newblock ``Assessment of Tools and Data for System-Level Dynamic Analyse,''
\newblock , INL/EXT-11 (2011).

\bibitem{rasmussen2006gaussian}
C.~E. RASMUSSEN and C.~K. WILLIAMS,
\newblock {\em Gaussian processes for machine learning},
\newblock The MIT Press (2006).

\bibitem{muller}
M.~M\"ULLER,
\newblock ``Dynamic Time Warping,''
\newblock in {\em Information Retrieval for Music and Motion}, pages 69--84,
  Springer Berlin Heidelberg, 2007.

\bibitem{wigeland2014nuclear}
R.~WIGELAND et~al.,
\newblock ``Nuclear Fuel Cycle Evaluation and Screening—Final Report,''
\newblock {\em USA: US DOE Fuel Cycle Technologies}, {\bf INL/EXT-14-31465}
  (2014).

\bibitem{mouginot2015ofcb}
B.~MOUGINOT,
\newblock ``Open Fuel Cycle Benchmark,''
\newblock public communication, 2015.

\bibitem{hodlr}
S.~{Ambikasaran}, D.~{Foreman-Mackey}, L.~{Greengard}, D.~W. {Hogg}, and
  M.~{O'Neil},
\newblock ``{Fast Direct Methods for Gaussian Processes and the Analysis of
  NASA Kepler Mission Data},''
\newblock (2014).

\bibitem{rasmussen1996evaluation}
C.~E. RASMUSSEN,
\newblock {\em Evaluation of Gaussian processes and other methods for
  non-linear regression},
\newblock PhD thesis, Citeseer, 1996.

\bibitem{ko2009gp}
J.~KO and D.~FOX,
\newblock ``GP-BayesFilters: Bayesian filtering using Gaussian process
  prediction and observation models,''
\newblock {\em Autonomous Robots}, {\bf 27}, 75 (2009).

\bibitem{myers1980performance}
C.~MYERS, L.~R. RABINER, and A.~E. ROSENBERG,
\newblock ``Performance tradeoffs in dynamic time warping algorithms for
  isolated word recognition,''
\newblock {\em Acoustics, Speech and Signal Processing, IEEE Transactions on},
  {\bf 28}, 623 (1980).

\bibitem{muda2010voice}
L.~MUDA, M.~BEGAM, and I.~ELAMVAZUTHI,
\newblock ``Voice recognition algorithms using mel frequency cepstral
  coefficient (MFCC) and dynamic time warping (DTW) techniques,''
\newblock {\em arXiv preprint arXiv:1003.4083} (2010).

\bibitem{yacout2005modeling}
A.~YACOUT, J.~JACOBSON, G.~MATTHERN, S.~PIET, and A.~MOISSEYTSEV,
\newblock ``Modeling the Nuclear Fuel Cycle,''
\newblock {\em Proc. The 23rd International Conference of the System Dynamics
  Society," Boston}, 2005.

\bibitem{feng2015dymond}
B.~FENG,
\newblock ``DYMOND EG01 to EG23 data,''
\newblock private communication.

\bibitem{DBLP:journals/corr/HuffGCFMOSSW15}
K.~D. HUFF et~al.,
\newblock ``Fundamental Concepts in the Cyclus Fuel Cycle Simulator
  Framework,''
\newblock {\em CoRR}, {\bf abs/1509.03604} (2015).

\bibitem{cyclus_v1_0}
R.~W. CARLSEN et~al.,
\newblock ``{Cyclus v1.0.0},''
\newblock (2014),
\newblock http://dx.doi.org/10.6084/m9.figshare.1041745.

\bibitem{scikit-learn}
F.~PEDREGOSA et~al.,
\newblock ``Scikit-learn: Machine Learning in {P}ython,''
\newblock {\em Journal of Machine Learning Research}, {\bf 12}, 2825 (2011).

\bibitem{merletti1999standards}
R.~MERLETTI and P.~DI~TORINO,
\newblock ``Standards for reporting EMG data,''
\newblock {\em J Electromyogr Kinesiol}, {\bf 9}, 3 (1999).

\bibitem{moreland2003fft}
K.~MORELAND and E.~ANGEL,
\newblock ``The FFT on a GPU,''
\newblock {\em Proc. Proceedings of the ACM SIGGRAPH/EUROGRAPHICS conference on
  Graphics hardware}, pages 112--119, Eurographics Association, 2003.

\bibitem{imai1983cepstral}
S.~IMAI,
\newblock ``Cepstral analysis synthesis on the mel frequency scale,''
\newblock {\em Proc. Acoustics, Speech, and Signal Processing, IEEE
  International Conference on ICASSP'83.}, volume~8, pages 93--96, IEEE, 1983.

\bibitem{mcfee2015librosa}
B.~MCFEE et~al.,
\newblock ``librosa: Audio and music signal analysis in python,''
\newblock {\em Proc. Proceedings of the 14th Python in Science Conference},
  2015.

\bibitem{milner2002speech}
B.~MILNER and X.~SHAO,
\newblock ``Speech reconstruction from mel-frequency cepstral coefficients
  using a source-filter model.,''
\newblock {\em Proc. INTERSPEECH}, Citeseer, 2002.

\bibitem{sato2007emotion}
N.~SATO and Y.~OBUCHI,
\newblock ``Emotion recognition using mel-frequency cepstral coefficients,''
\newblock {\em Information and Media Technologies}, {\bf 2}, 835 (2007).

\bibitem{EBUTECH3285}
EBU-UBR,
\newblock ``Specification of the Broadcast Wave Format (BWF) - A format for
  audio data files in broadcasting version 2.0,''
\newblock  EBU-TECH 3285, EUROPEAN BROADCASTING UNION (2011).

\bibitem{juang1991hidden}
B.~H. JUANG and L.~R. RABINER,
\newblock ``Hidden Markov models for speech recognition,''
\newblock {\em Technometrics}, {\bf 33}, 251 (1991).

\end{thebibliography}

\end{document}
